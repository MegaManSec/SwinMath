\documentclass{article}
\usepackage[utf8]{inputenc}
\usepackage[english]{babel}
\usepackage[]{amsthm}
\usepackage[]{amssymb}
\usepackage[]{amsmath}
\usepackage[]{hyperref}
\usepackage[]{xcolor}
\usepackage[]{fancyhdr}
\hypersetup{
    colorlinks,
    linkcolor={red!50!black},
    citecolor={blue!50!black},
    urlcolor={blue!80!black}
}


\pagestyle{fancy}
\fancyhf{}
\title{\vspace{-4cm}MTH30002 - Differential Equations \\Assignment 4}
\author{Joshua Rogers}
\lhead{MTH30002 Assignment 4}
\rhead{Joshua Rogers 101096819}
\date\today

\begin{document}
\maketitle 

\section*{Question 3.L}
Given the Legendre's equation
\begin{equation}\label{eq:3L}
y^{''}-\frac{2x}{1-x^2}y^{'}+\frac{n(n+1)}{1-x^2}y=0
\end{equation}
find the Taylor series expansion for $\frac{1}{1-x^{2}}$ near $x_0 = 0$, and thus show that equation \textbf{\ref{eq:3L}} is analytical at $x_0 = 0$.


\subsection*{Answer}

A function $f$ is called analytic at $x_0$ if it is given by a convergent power series.

Consider the Taylor series expansion of $f(x) = \frac{1}{1-x^2}$ around $x_0 = 0$.

\begin{equation}
f(x) = f(x_0) + \frac{f^{'}(x_0)(x-x_0)}{1!} + \frac{f^{''}(x_0)(x-x_0)^{2}}{2!} + ... + \frac{f^{n}(x_0)(x-x_0)^{n}}{n!}
\end{equation}

\begin{align*}
&f(0) = 1\\
&f^{'}(x) = \frac{2x}{(1-x^2)^2} \therefore f^{'}(x)x = 0\\
&f^{''}(x) = \frac{8x^2}{(1-x^2)^3}+\frac{2}{(1-x)^2} \therefore \frac{f^{''}(0)x^2}{2!} = x^2\\
&f^{'''}(x) = \frac{24x}{(1-x^2)^3} + \frac{48x^3}{(1-x^2)^4} \therefore f^{'''}(0)x^3 = 0\\
&f^{''''}(x) = \frac{288x^2}{(1-x^2)^4}+\frac{24}{(1-x^2)^3}+\frac{384x^4}{(1-x^2)^5} \therefore \frac{f^{''''}(0)x^4}{4!} = x^4
\end{align*}



Therefore, the Taylor series about $x_0 = 0$ is clearly $\sum_{n=0}^{\infty} x^{2n}$.

\begin{equation}\label{limits1}
L = \lim_{n\to\infty} \left\| \frac{a_{n+1}}{a_n} \right\| = \lim_{n\to\infty}\left\|\frac{x^{2(n+1)}}{x^{2n}}\right\| = \left|x^2\right|. 
\end{equation}

\begin{align*}
R =& \left\|x^2\right\| < 1 \\
& \sqrt{\left\|x^2\right\|} < \sqrt{1}\\
\\
R =& \left\|x\right\| < 1
\end{align*}

The series is by definition convergent.

Continuing...
\begin{equation*}
\frac{-2x}{1-x^2} = -2x\cdot \sum_{n=0}^{\infty} {x^{2n}} = -2\cdot \sum_{n=0}^{\infty} {x^{2n+1}} \text{  around } x_0=0.
\end{equation*}

\begin{equation}\label{limits2}
L = \lim_{n\to\infty} \left\| \frac{a_{n+1}}{a_n} \right\| = \lim_{n\to\infty}\left\|\frac{x^{2(n+1)+1}}{x^{2n+1}}\right\| = \left|x^2\right|.
\end{equation}
    
\begin{align*}
R =& \left\|x^2\right\| < 1 \\
& \sqrt{\left\|x^2\right\|} < \sqrt{1}\\
\\
R =& \left\|x\right\| < 1
\end{align*}

The series is also convergent.

\begin{equation}\label{limits3}
\frac{n(n+1)}{1-x^2} = n(n+1) \sum_{m=0}^{\infty} x^{2m} \text{  around } x_0=0.
\end{equation}

Clearly, at $x_0 = 0$, the equation \textbf{\ref{eq:3L}} is analytic as shown in \textbf{\ref{limits1}}, \textbf{\ref{limits2}}, and \textbf{\ref{limits3}}.

\section*{Question 3.M}
Using the recurrence equation
\begin{align}\label{3M}
a_{s+2} = \frac{-(n-s)(n+s+1)\cdot{a_s}}{(s+2)(s+1)}\Bigr|
s\geq0
\end{align}
show that the radius of convergence R = 1.

\subsection*{Answer}


For $a_1 = 0$ all the odd terms in equation \ref{3M} are 0. Thus,  $b_s = a_{2s}$.
Then:
\begin{align*}
y = \sum_{n=0}^{\infty} b_s x^{2s}\\\\
b_{s+1} =  \frac{-(n-2s)(n+2s+1)}{(2s+2)(2s+1)}
\end{align*}

Performing the ratio test:
\begin{align*}
\lim_{s\to\infty}\left| \frac{-(n-2s)(n+2s+1)}{(2s+2)(2s+1)} x^{2s+2} \cdot \frac{(s+2)(s+1)}{-(n-s)(n+s+1)x^{2s}}\right| = \left|x^2\right|.\\
\left|x^2\right| < 1\\
\left|x\right| < \sqrt{1}
\end{align*}

Therefore, $R = 1$.

\end{document}
