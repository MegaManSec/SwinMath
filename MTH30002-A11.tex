\documentclass{article}
\usepackage[utf8]{inputenc}
\usepackage[english]{babel}
\usepackage[]{amsthm}
\usepackage[]{amssymb}
\usepackage[]{amsmath}
\usepackage[]{hyperref}
\usepackage[]{cancel}
\usepackage[]{graphicx}
\usepackage[]{xcolor}
\usepackage[]{tabularx}
\usepackage[]{fancyhdr}
\hypersetup{
    colorlinks,
    linkcolor={red!50!black},
    citecolor={blue!50!black},
    urlcolor={blue!80!black}
}

\pagestyle{fancy}
\fancyhf{}
\title{\vspace{-4cm}MTH30002 - Differential Equations Assignment 11}
\author{Joshua Rogers}
\lhead{MTH30002 Assignment 11}
\rhead{Joshua Rogers 101096819}
\date\today

\begin{document}
\maketitle

\section*{Exercise 5.3}

We have the boundary condition

$$u(1,\theta) = f(\theta)$$

We actually have

$$
u(1,\theta) = \begin{cases}
0 & -\pi < \theta \leq -\frac{\pi}{2} \\
110 & -\frac{\pi}{2} < \theta < \frac{\pi}{2} \\
0 & \frac{\pi}{2} \leq \theta < \pi \\
   \end{cases}
$$

$$u(r,\theta) = F(r)G(\theta)$$

placing this into the circular Laplace's equation and setting the solution to $K$ we get

\begin{align*}
r^2 F^{''} + rF^{'}-KF=0 \\
KG+G^{''}=0
\end{align*}


For $K \geq 0$, we let $K=k^2$

$$G(\theta) = Acos(k\theta) + Bsin(k\theta)$$

We have the boundary conditions

$$G\left(\frac{\pi}{2}\right) = G\left(\frac{3\pi}{2}\right) = 0$$

$$\det \left( \begin{bmatrix} cos\left(k\frac{\pi}{2}\right) & sin\left(k\frac{\pi}{2}\right) \\ cos\left(k\frac{-\pi}{2}\right) & sin\left(k\frac{-\pi}{2}\right) \end{bmatrix} \right) = 0 $$

This is clearly true for all integer values of $k$.

Thus, we have the eigenfunction

$$G_k(\theta) = A_k cos(k\theta) + B_k sin(k\theta)$$



Similarly, we have for $K \geq 0$, we let $K=k^2$
$$F(r) = A + B \ln r$$
and
$$F(r) = \tilde{A} r^k + \tilde{B} r^{-k}$$
To satisfy the condition that $|F(r)| < \infty$ we must have $B = \tilde{B} = 0$

Thus, we have

$$F(r) = A k^p\Bigr| p \geq 0$$
Thus, for inside the disk, we have

$$u(r,\theta) = \sum r^n G_n(\theta) = \sum_{n=0}^{\infty} A_n r^n cos(n \theta) + \sum_{n=1}^{\infty} B_n r^n sin (n\theta)$$

Applying our boundary conditions and given the fact this is clearly a fourier series, we have

$$A_0 = \frac{1}{2\pi} \int_{-\pi/2}^{\pi/2} 110 d\theta = 55$$

$$A_n= \frac{1}{\pi} \int_{-\pi/2}^{\pi/2} 110 cos(n\theta) d\theta$$
$$A_n = \frac{220}{n \pi} sin\left(\frac{n\pi}{2}\right)$$

Likewise we have $B_n = 0$

Thus, as $A_n=0$ for $n=2n-1$, we have

$$u(r,\theta) = 55 -220 \sum_{n=1}^{\infty} r^{2n-1} \left(\frac{(-1)^n}{(2n-1)\pi}\right) cos((2n-1)\theta)$$



\section*{Exercise 5.4}
The motion equation for a drum is given by
$$
w_{xx}+w_{yy}=\frac{1}{c^2}w_{tt}
$$

thus we have the general solution

$$
U_{mn}(x,y,t) = \left(B_{mn}cos(\lambda_{mn}t)+B^{*}_{mn}sin\left(\lambda_{mn}t)\right)\right)sin\left(\frac{m \pi x}{a}\right)sin\left(\frac{n \pi y}{b}\right)
$$

where

$$
\lambda_{mn} = c\pi \sqrt{\mu_a^2 + \mu_b^2}\Bigr| m,n \geq 1
$$

$$
u_{mn} = \frac{\lambda_{mn}}{2\pi }
$$

thus
$$
f = \frac{c}{2} \sqrt{\mu_a^2 + \mu_b^2}
$$

By defition, we have 
\begin{align*}
c^2 &= \frac{T}{p}\\
c &= \sqrt{\frac{T}{p}}
\end{align*}

By doubling the tension variable $T$ we have

\begin{align*}
f_2 = \frac{c}{2} \sqrt{\mu_a^2 + \mu_b^2} \\
f_2 = \frac{1}{2} \sqrt{\frac{2T}{p}}\sqrt{\mu_a^2 + \mu_b^2} \\
f_2 = \frac{1}{2} \sqrt{2} \sqrt{\frac{T}{p}}\sqrt{\mu_a^2 + \mu_b^2} \\
f_2 = \sqrt{2} f_1
\end{align*}

Thus, the frequency is multiplied by $\sqrt{2}$ when the tension of the membrane is doubled.

\section*{Exercise 5.5}

No. The Bessel function $J_0(x)$ in the equation with frequency $\frac{\lambda_{mn}}{2\pi}$ will not be able to have two different nodal lines corresponding to the same frequency.
The equation indicates that different nodal lines will have different radiuses for different eigenfunctions, as $\frac{\lambda_{mn}}{2\pi}$ will be different for different eigenfunctions, and because $J_0(x)$ is not reguarly spaced on the axis (thus meaning the 'function' cannot be expressed as the sum of two squares). (I'd love to show the cool photo on P589 in Kreyzig to show how it's not possible but I don't want to include the pic verbatim in here.)



\end{document}
