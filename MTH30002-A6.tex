
\documentclass{article}
\usepackage[utf8]{inputenc}
\usepackage[english]{babel}
\usepackage[]{amsthm}
\usepackage[]{amssymb}
\usepackage[]{amsmath}
\usepackage[]{hyperref}
\usepackage[]{cancel}
\usepackage[]{xcolor}
\usepackage[]{fancyhdr}
\hypersetup{
    colorlinks,
    linkcolor={red!50!black},
    citecolor={blue!50!black},
    urlcolor={blue!80!black}
}

\pagestyle{fancy}
\fancyhf{}
\title{\vspace{-4cm}MTH30002 - Differential Equations \\Assignment 6}
\author{Joshua Rogers}
\lhead{MTH30002 Assignment 6}
\rhead{Joshua Rogers 101096819}
\date\today

\begin{document}
\maketitle

\section*{Problem 11}

Solve the following equation:

\begin{equation}\label{11} \left(\frac{y'}{x}\right)^{'} + \frac{\left(\lambda + 1\right)y}{x^3} = 0\end{equation}

for $y(1) = 0$ and $y(e^\pi)=0$
\subsection*{Answer}

For $\lambda > 0$ we let $\lambda = k^2$.
Substituting this into \eqref{11} produces
$$ \left(\frac{y'}{x}\right)^{'} + \frac{\left(k^2 + 1\right)y}{x^3} = 0 $$

Using the Euler-Couchy method, we obtain the solution

$$ y= \left[ c_1 x^{ik} + c_2 x^{-ik}\right] \cancel{x} $$
thus, we obtain
\begin{equation}\label{2}y = c_1 cos\left(k \ln x \right) + c_2 sin\left( k \ln x \right)\end{equation}

Plugging in the endpoint $y(1)=0$ into \eqref{2}, we find that $c_1=0$ and thus
\begin{equation}\label{25}
y = c_2 sin\left(k \ln x \right)\Bigr| c_2 \neq 0
\end{equation}

For an eigenvalue of equation $\eqref{11}$, with $y(1) = 0$ and $y(e^\pi) = 0$, the equation must satisfy the following condition.

$$
\det \left( \begin{bmatrix} 1 & 0 \\ cos(k\pi) & sin(k\pi) \end{bmatrix} \right) = 0
$$

We obtained this as $\cos(0) = 1$ and $\sin(0) = 0$.

Solving, we obtain $sin(k\pi) = 0$.

The sin function returns 0 for integer values of $\pi$.

Thus, we obtain
\begin{align*}k\pi = n\pi \\
k = n
\end{align*}

Thus, we have the eigenvalues:
$$\lambda_{n} = k^2 = n^2$$

Given $\eqref{25}$ the eigenfunctions are
$$ y_n = sin \left( n\cdot \ln x \right ) $$.


\section*{Problem 9}

Solve the following equation:

\begin{equation}\label{9} y^{''} + \lambda y =0\end{equation}

for $y(0) = 0$ and $y^{'}(L)=0$
\subsection*{Answers}

For $\lambda >0$ we have the solution $y = c_1 cos\left(\sqrt \lambda x \right) + c_2 sin\left( \sqrt \lambda x \right)$

Given the endpoint $y(0) = 0$ we find that $c_1 =0$ and thus we solve for the solution.

\begin{align}
    &y = c_2 sin\left( \sqrt \lambda x \right) \Bigr| c_2 \neq 0  \label{6}  \\
    &y' = c_2 \sqrt \lambda \cdot cos \left( \sqrt \lambda x \right) \Bigr| c_2 \neq 0 \nonumber
\end{align}

Given the endpoint $y^{'}(L)=0$ we obtain
\begin{align*}
&0 = c_2 \cdot cos \left( \sqrt \lambda L \right) \\
&0 = cos\left ( \sqrt \lambda L \right)
\end{align*}
$cos(x)$ is zero for values $x = \frac{\pi}{2}, \frac{3\pi}{2}, \cdots , \frac{(2n-1) \pi}{2}$ thus we find the eigenvalues

\begin{align}
    &\sqrt \lambda = \frac{(2n-1)\pi}{2L} \nonumber \\
    & \lambda_{n} = \frac{(2n-1)^2 \pi^2}{4L^2} \label{4}
\end{align}

Given equation $\eqref{4}$ and $\eqref{6}$, we find the eigenfunctions.

$$
y_n = sin \left ( \frac{(2n-1)\pi\cdot x}{2L} \right)
$$
\end{document}
