

\documentclass{article}
\usepackage[utf8]{inputenc}
\usepackage[english]{babel}
\usepackage[]{amsthm}
\usepackage[]{amssymb}
\usepackage[]{amsmath}
\usepackage[]{hyperref}
\usepackage[]{cancel}
\usepackage[]{xcolor}
\usepackage[]{fancyhdr}
\hypersetup{
    colorlinks,
    linkcolor={red!50!black},
    citecolor={blue!50!black},
    urlcolor={blue!80!black}
}

\pagestyle{fancy}
\fancyhf{}
\title{\vspace{-4cm}MTH30002 - Differential Equations Assignment 6}
\author{Joshua Rogers}
\lhead{MTH30002 Assignment 6}
\rhead{Joshua Rogers 101096819}
\date\today

\begin{document}
\maketitle

\section*{Problem 11}

Solve the following equation:

\begin{align}\label{11}
\left(\frac{y^{'}}{x}\right)^{'} + \frac{\left(\lambda + 1\right)y}{x^3} = 0 && &y(1) =0 \\
&& &y(e^\pi)=0 \nonumber
\end{align}
\subsection*{Answer}

\subsubsection*{Case $\lambda=0$}
$$ \left(\frac{y^{'}}{x}\right)^{'} + \frac{y}{x^3} = 0 $$

Using the Euler-Couch method, we obtain
$$y = c_1 x + c_2 x \ln x$$

For $y=0$ and $x=1$ we find $c_1 = 0$.

For $y=0$ and $x=e^\pi$ we find $c_1+c_2\pi=0$.

$$\det \left( \begin{bmatrix} 1 & 0 \\ 1 & \pi \end{bmatrix} \right) \neq 0$$

Therefore, we have no eigenvalues for this case, only the trivial solution $y=0$. This is because as the determinate is not 0, the set of equations are not orthogonal, which they must be to satisfy Sturm-Liouville problems.

\subsubsection*{Case $\lambda < 0$}
We allow $\lambda = -k^2$.
$$\left(\frac{y^{'}}{x}\right)^{'} + \frac{\left(1-k^2 \right)y}{x^3} = 0$$

$$y = \left[c_1x^{-1}+c_2x\right]x^k$$

For $y=0$ and $x=1$ we find $c_1+c_2=0$

For $y=0$ and $x=e^\pi$ we find $e^{-\pi}c_1 + e^{\pi}c_2 = 0$

$$\det \left( \begin{bmatrix} 1 & 1 \\ e^{\pi} & e^{-\pi} \end{bmatrix} \right) \neq 0$$
Therefore, we have no eigenvalues for this case, only the trivial solution $y=0$.

\subsubsection*{Case $\lambda > 0$}

We allow $\lambda = k^2$.
Substituting this into \eqref{11} produces
$$ \left(\frac{y^{'}}{x}\right)^{'} + \frac{\left(k^2 + 1\right)y}{x^3} = 0 $$

Using the Euler-Couchy method, we obtain the solution

$$ y= \left[ c_1 x^{ik} + c_2 x^{-ik}\right] x $$
thus, we obtain
\begin{equation}\label{2}y = x c_1 cos\left(k \ln x \right) + x c_2 sin\left( k \ln x \right)\end{equation}

Plugging in the endpoint $y(1)=0$ into \eqref{2}, we find that $c_1=0$ and thus
\begin{equation}\label{25}
y = x c_2 sin\left(k \ln x \right)\Bigr| c_2 \neq 0
\end{equation}

For an eigenvalue of equation $\eqref{11}$, with $y(1) = 0$ and $y(e^\pi) = 0$, the equation must satisfy the following condition.

$$
\det \left( \begin{bmatrix} 1 & 0 \\ cos(k\pi) & sin(k\pi) \end{bmatrix} \right) = 0
$$

We obtained this as $\cos(0) = 1$ and $\sin(0) = 0$.

Solving, we obtain $sin(k\pi) = 0$.

The sin function returns 0 for integer values of $\pi$.

Thus, we obtain
\begin{align*}k\pi = n\pi \\
k = n
\end{align*}

Thus, we have the eigenvalues:
$$\lambda_{n} = k^2 = n^2$$

Given $\eqref{25}$ the eigenfunctions are
$$ y_n = x sin \left( n\cdot \ln x \right ) $$.

\pagebreak


\section*{Problem 9}

Solve the following equation:

\begin{align}\label{9}
y^{''} + \lambda y =0 && &y(0) = 0 \\
&& &y^{'}(L) =0 \nonumber
\end{align}

\subsection*{Answers}
\subsubsection*{Case $\lambda=0$}
$$ y^{''}=0 $$

Thus, clearly,
$$y^{'} = c_1$$
$$y = c_1 x + c_2 $$

For $y=0$ and $x=0$ we find $c_2 = 0$.

For $y^{'}=0$ and $x=L$ we find $c_1=0$.

$$\det \left( \begin{bmatrix} 0 & 1 \\ 1 & 0 \end{bmatrix} \right) \neq 0$$

Therefore, we have no eigenvalues for this case, only the trivial solution $y=0$.

\subsubsection*{Case $\lambda < 0$}
We allow $\lambda = -k^2$.
\begin{align*}
y^{''}-k^{2}y=0 \\
y = c_1 e^{kx} + c_2 e^{-kx} \\
y^{'} = \left[c_1 e^{kx} - c_2 e^{-kx}\right] k
\end{align*}

For $y=0$ and $x=0$ we find $c_1+c_2=0$

For $y^{'}=0$ and $x=L$ we find $c_1 e^{kL} - c_2 e^{-kL} = 0$

Letting $c_1 = -c_2$, we obtain $c_1\left[ e^{kL} +e^{-kL}\right] = 0$

$$\det \left( \begin{bmatrix} 1 & 1 \\ e^{kL} & e^{-kL} \end{bmatrix} \right) \neq 0\Bigr|  kL\in\mathbb{N} $$
Therefore, we have no eigenvalues for this case, only the trivial solution $y=0$.





\subsubsection*{Case $\lambda > 0$}

We allow $\lambda = k^2$.
\begin{align*}
y^{''}+k^{2}y=0 \\
y = c_1 cos\left(kx\right) + c_2 sin\left(kx\right) \\
y^{'} = \left[-c_1 sin\left(kx\right) + c_2 cos\left(kx\right) \right] k
\end{align*}

For $y=0$ and $x=0$ we find $c_1 = 0$.

For $y^{'}=0$ and $x=L$ we find $c_2 \cdot k \cdot cos(kL) = 0$.
\begin{equation}\label{matrix2}
\det \left( \begin{bmatrix} 1 & 0 \\ 0 & cos(kL) \end{bmatrix} \right) = 0
\end{equation}

To satisfy the matrix $\eqref{matrix2}$, we must solve for values of $k$ such that $cos\left(kL\right) = 0$.

\begin{align*}
    kL = \frac{\left(2n+1\right)\pi}{2} \\
    k = \frac{\left(2n+1\right)\pi}{2L}
\end{align*}

Thus, our eigenvalues are

$$\lambda_n = k^2 = \frac{\left(2n+1\right)^2\pi^2}{4L^2}$$

and given $c_1=0$, we find our eigenfunction

$$y_n = sin\left( k \cdot x\right) = sin \left( \frac{\left(2n+1\right)\pi \cdot x}{2L}\right)$$


\pagebreak
\section*{Comments}
Using the fact that each pair of equations have a determinate of 0, we thus conclude that the solutions obtained are orthogonal. 

Or, we use the following integrals.


\subsection*{Problem 9}

$$\int_0^{L} sin\left(\frac{n \pi x}{L}\right) \cdot sin\left(\frac{m \pi x}{L}\right) = \frac{1}{2} \int_0^{L} cos \left(\frac{\pi\left(n-m\right)x}{L}\right) - cos\left(\frac{\pi\left(n+m+1\right)x}{L}\right)$$
This integral has the solution
$$\frac{1}{2} \left( \frac{L sin(\pi(z_1))}{\pi(z_1)} - \frac{L sin(\pi(z_2))}{\pi(z_2)}\right)$$
where $z_1 = n-m$ and $z_2 = n+m+1$.

For all integer values of both $z_1$ and $z_2$, the integral evaluates to 0 as sin returns 0 for all integer values of $\pi$ and 0, therefore confirming orthogonality.

\subsubsection*{Problem 11}


$$\int_1^{e^\pi} \frac{1}{x^3} \cdot xsin\left(n \ln x\right) \cdot xsin\left(m \ln x\right) dx$$

We let $u = \ln x$ and $du = \frac{1}{x}$ and $z_1 = n-m$ and $z_2 = n+m$, thus

\begin{align*}
&\frac{1}{2} \int_0^{\pi} cos\left(u\left[z_1\right]\right)-cos\left(u\left[z_2\right]\right) du \\
&\frac{1}{2} \left[ \frac{sin\left(u\left(z_1\right)\right)}{z_1}-\frac{sin\left(u\left(z_2\right)\right)}{z_2}\right]\Bigr|_0^{\pi} \\
&\frac{1}{2} \left[ \frac{sin\left(\pi\left(z_1\right)\right)}{z_1} - \frac{sin\left(\pi\left(z_2\right)\right)}{z_2}\right] = 0 \Bigr| z_1, z_2 \in\mathbb{R}
\end{align*}
Clearly, for all integers of both $z_1$ and $z_2$, the answer will be 0, thus proving orthogonality.

\end{document}
