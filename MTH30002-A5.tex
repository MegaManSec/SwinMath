\documentclass{article}
\usepackage[utf8]{inputenc}
\usepackage[english]{babel}
\usepackage[]{amsthm}
\usepackage[]{amssymb}
\usepackage[]{amsmath}
\usepackage[]{hyperref}
\usepackage[]{xcolor}
\usepackage[]{fancyhdr}
\hypersetup{
    colorlinks,
    linkcolor={red!50!black},
    citecolor={blue!50!black},
    urlcolor={blue!80!black}
}

\pagestyle{fancy}
\fancyhf{}
\title{\vspace{-4cm}MTH30002 - Differential Equations Assignment 5}
\author{Joshua Rogers}
\lhead{MTH30002 Assignment 5}
\rhead{Joshua Rogers 101096819}
\date\today

\begin{document}
\maketitle 

\section*{Question 3.Q}

Find the appropriate solutions to the following ODEs:

\begin{equation}\label{34a} (x+2)^2y^{''}-2y=0 \end{equation}
\begin{equation}\label{34b} 4xy^{''}+2^{'}+y=0 \end{equation}
\begin{equation}\label{34c} 2xy^{''}-(8x-1)y^{'}+(8x-2)y=0 \end{equation}
\begin{equation}\label{34d} x^2y^{''}+4xy^{'}-(x^2-2)y=0 \end{equation}
\subsection*{Answers}

\subsubsection*{3.4 A}
We may solve \textbf{\eqref{34a}} using substitution.
\begin{align*}
y &= \frac{u}{(x+2)}\\
y^{''} &= \frac{u^{''}}{(x+2)}-\frac{2u^{'}}{(x+2)^2}+\frac{2u}{(x+2)^3}
\end{align*}

\begin{align*}
(x+2)^2 \left( \frac{u^{''}}{(x+2)}-\frac{2u^{'}}{(x+2)^2}+\frac{2u}{(x+2)^3} \right) - \frac{2u}{(x+2)} &= 0 \\
(x+2)u^{''}-2u^{'} &= 0
\end{align*}

We let $w=u^{'}$ and $w^{'}=u^{''}$

\begin{align*}
(x+2) w^{'} - 2w = 0.\\
w^{'} = \frac{2w}{(x+2)}\\
\int{\frac{w^{'}}{w}} = \int{\frac{2}{(x+2)}}\\
\ln{\left|w\right|} = 2\ln{\left|x+2\right|}+C\\\\
\therefore w(x) = (\left|x+2\right|)^2 \cdot e^c
\end{align*}

We allow $e^c = c_1$ and therefore have
$$ w(x) = c_1(\left|x+2\right|)^2 $$
$|(x+2)|^2$ is always positive so we can remove the absolute signs. From our substitutions, we find $u(x)$ and thus $y(x)$.

\begin{align*}
\int{u^{'}} &= \int{c_1(x+2)^2}\\
u &= \frac{c_1(x+2)^3}{3} + c_2\\
u &= c_1(x+2)^3 + c_2 \Bigr|c_1 = 3\\\\
y &= \frac{u}{x+2}
\end{align*}

Here, we find the solution to the inital ODE.
\[
y(x) = c_1(x+2)^2 + \frac{c_2}{(x+2)}
\]

\subsubsection*{3.4 B, 3.4 C, 3.4 D}
Solving \textbf{\eqref{34b}}, \textbf{\eqref{34c}}, and \textbf{\eqref{34d}}, we use the Frobenius Method.

First we double check that the Taylor series expansion exists for each $p(x) \cdot x$ and $q(x) \cdot x^2$ at $x_0=0$; they do, therefore we can proceed with Forebius' method.


We allow the following solutions:
\begin{align}\label{frobenius}
y &= \sum_{n=0}^{\infty}{a_n\cdot x^{n+r}}\\
y^{'} &= \sum_{n=0}^{\infty}{(n+r) \cdot a_n \cdot x^{n+r-1}}\\
y^{''} &= \sum_{n=0}^{\infty}{(n+r)(n+r-1) \cdot a_n \cdot x^{n+r-2}}
\end{align}

Solving for \textbf{\eqref{34b}}, we check the indicial equation $r(r-1)+\frac{1}{2}r = 0$ therefore $r =0,1/2$.

\begin{align*}
4\sum_{n=0}^{\infty}(n+r-1)(n+r) \cdot a_n \cdot x^{n+r-1} + 2\sum_{n=0}^{\infty}(n+r) \cdot a_n \cdot x^{n+r-1} + \sum_{n=0}^{\infty} \cdot a_n \cdot x^{n+r} &= 0 \\
4\sum_{n=-1}^{\infty}(n+r)(n+1+r) \cdot a_{n+1} \cdot x^{n+r} + 2\sum_{n=-1}^{\infty} (n+1+r) \cdot a_{n+1} \cdot x^{n+r} + \sum_{n=0}^{\infty} a_n \cdot x^{n+r} &= 0
\end{align*}

By allowing $n=-1$ for the first two sums, we once again identify the indical equation $4(r-1)(r) + 2(r) = 0)$ or $r=0,\frac{1}{2}$ as expected.

By factoring out $x^{n+r}$ and equations the sums beginning from $n=0$ due to linear independence, we find the recurrence relation:
\[
\left[4(n+r)(n+1+r)+2(n+1+r)\right] \cdot a_{n+1} + a_n = 0
\]
\begin{equation}\label{34be}
a_{n+1} = \frac{-a_n}{4(n+r)(n+r+1)+2(n+r+1)} \Bigr| s\geq0
\end{equation}


Using the recurrence relation \textbf{\eqref{34be}} and letting $r=\frac{1}{2}$ we find:
\begin{align*}
n=0, a_1 &= \frac{-a_0}{3!}\\
n=1, a_2 &= \frac{-a_1}{5 \cdot 4} = \frac{a_0}{5!}\\
n=2, a_3 &= \frac{-a_2}{7 \cdot 6} = \frac{-a_0}{7!}
\end{align*}
From inspection, we can find that
\begin{equation}\label{34ba}
a_n = \frac{(-1)^n}{(2n+1)!}
\end{equation}
By induction (i.e. wolframalpha) we can find our first solution to the ODE. Placing the $a_n$ from \textbf{\eqref{34ba}} and $ r = \frac{1}{2} $ in equation \textbf{\eqref{frobenius}}, we find:
\[
\sum_{n=0}^{\infty}{\frac{(-1)^n}{(2n+1)!} \cdot x^{n+\frac{1}{2}}} = sin(\sqrt{x})
\]
As such, our first solution is:
\[y_1 = c_1sin(\sqrt{x})
\]

Now for the second solution, we use the recurrence relation \textbf{\eqref{34be}} and let $r=0$.
\[
a_{n+1} = \frac{-a_n}{(2n+2)(2n+1)} \Bigr| s\geq0
\]

\begin{align*}
n=0, a_1 &= \frac{-a_0}{2}\\
n=1, a_2 &= \frac{-a_1}{4 \cdot 3} = \frac{a_0}{4!}\\
n=2, a_3 & = \frac{-a_2}{6 \cdot 5} = \frac{-a_0}{6!}
\end{align*}
Allowing $a_0=1$ we find $a_n = \frac{(-1)^n}{(2n)!}$.
By induction we find the second solution to the ODE.
\[
\sum_{n=0}^{\inf} x^{n-0} \cdot \frac{(-1)^n}{(2n)!} = cos(\sqrt{x})
\]
\[
y(x) = c_1 \cdot sin(\sqrt{x}) + c_2 \cdot cos(\sqrt{x})
\]
\par

Solving for \textbf{\eqref{34c}}, we first find our expected values of r: $r(r-1)+\frac{1}{2} =0$, $r = 0,\frac{1}{2}$

\begin{align*}
2\sum_{n=0}^{\infty}(n+r)(n+r-1) \cdot a_n \cdot x^{n+r-1} + \sum_{n=0}^{\infty} (n+r) \cdot (n+r) \cdot x^{n+r-1} - 8\sum_{n=0}^{\infty} (n+r) \cdot a_n \cdot x^{n+r}\\
+ 8 \sum_{n=0}^{\infty} \cdot a_n \cdot x^{n+r+1} - 2 \sum_{n=0}^{\infty} \cdot a_n \cdot x^{n+r} =0\\
\sum_{n=0}^{\infty} (n+r)(n+r-\frac{1}{2}) \cdot a_n \cdot x^{n+r-1} - 4 \sum_{n=1}^{\infty} (n-\frac{3}{4}+r) \cdot a_{n-1} \cdot x^{n+r-1} + 4\sum_{n=2}^{\infty} a_{n-2} \cdot x^{n-1} = 0
\end{align*}

Solving the first two sums for $n=0$ and $n=1$ yields $a_0=0$ and $a_1=2a_0$. Given that we know $a_0$ cannot be 0, it is a free variable. We will allow $a_0 = 1$ and therefore $a1 = 2$.

Letting $r = 0$, we find the recurrence formula:
$$a_n = \frac{-4a_{n-2}+4(n+r-\frac{3}{4})a_{n-1}}{(n+r)(n+r-\frac{1}{2})} \Bigr| n \geq 2$$
\begin{align*}
n=2, a_2 = \frac{ -4 + 10}{2 \cdot \frac{3}{2}} = 2\\
n=3, a_3 = \cdots = \frac{4}{3}\\
n=4, a_4 = \cdots = \frac{2}{3}
\end{align*}

$$a_n = \frac{2^n}{n!}$$
$$\sum_{n=0}^{\infty} \frac{2^n \cdot x^n}{n!} = e^{2x}.$$

We have found the first solution, $y(x) = e^{2x}$.

Allowing $r=\frac{1}{2}$ we find the recurrence relation:
$$a_n = \frac{-4a_{n-2} + 4(n-\frac{1}{4})a_{n-1}}{(n+\frac{1}{2}) \cdot n} \Bigr| n \geq 2$$
\begin{align*}
n=2, a_2 = 2\\
n=3, a_3 = \frac{4}{3}\\
n=4, a_4 = \frac{2}{3}
\end{align*}

Remembering that $a_0 = 1$, we find:
$$a_n = \frac{2^n}{n!}$$
$$\sum_{n=0}^{\infty} \frac{2^n}{n!} \cdot x^{n+\frac{1}{2}} = e^{2x} \cdot \sqrt{x}$$

We have now found the solution to the ODE.

\[
y(x) = c_1 e^{2x} + c_2 e^{2x} \cdot \sqrt{x}
\]

\par
Finally, we can solve for equation \textbf{\eqref{34d}}.

\[
\sum_{n=0}^{\infty} (n+r-1)(n+r) \cdot a_n \cdot x^{n+r} + 4\sum_{n=0}^{\infty} (n+r) \cdot a_n \cdot x^{n+r} - \sum_{n=2}^{\infty} a_n \cdot x^{n+r} + 2 \sum_{n=0}^{\infty} a_n \cdot x^{n+r} = 0
\]
$(r-1)r+4r+2=0$ therefore $r = -1, -2$.

For $r = -1$, we find $a_0 = 1$, $a_1 = 0$, and $a_n = \frac{a_{n-2}}{n(n+1)} \Bigr | n \geq 2$.
\begin{align*}
n=2, a_2 = \frac{1}{3!}\\
n=3, a_3 = 0\\
n=4, a_4 = \frac{1}{5!}
\end{align*}

We find
$$a_{2n} = \frac{1}{(2n+1)!}$$
$$\sum_{n=0}^{\infty} x^{-2} \cdot \frac{x^{2n+1}}{(2n+1)!} = x^{-2} \cdot sinh(x) $$
Thus we have found solution number one, $y(x) = x^{-2} sinh(x)$.

For $r = -2$, we find $a_n = \frac{a_{n-2}}{n(n-1)} \Bigr| n \geq 2$

\begin{align*}
n=2, n_2 = \frac{1}{2!}\\
n=3, n_3 = 0\\
n=4, n_4 = \frac{1}{4!}\\
\end{align*}

We find
$$a_{2n} = \frac{1}{(2n)!}$$
$$\sum_{n=0}^{\infty} \frac{x^{2n-2}}{(2n)!} = x^{-2} \cdot cosh(x)$$

Thus, the solution to this ODE is given by the equation $$y(x) = c_1 \cdot x^{-2} \cdot sinh(x) + c_2 \cdot x^{-2} \cdot cosh(x)$$
\par
\section*{Question 3.S}
Reduce the following equations to Bessel's ODE and find a general solution.

\begin{equation}\label{3s3} xy^{''}+y^{'}+\frac{1}{4}y = 0 \end{equation}
\begin{equation}\label{3s8} (2x+1)^2y^{''} + 2(2x+1)y^{'} + 16x(x+1)y = 0 \end{equation}  
\begin{equation}\label{3s10} x^2y^{''}+(1-2v)xy^{'}+v^2(x^{2v}+1-v^2)y = 0 \end{equation}

\subsection*{Answers}

\subsubsection*{3.S.3}

For \textbf{\eqref{3s3}}, we allow $x=\sqrt{x}$ thus $x=z^2$.

\begin{align*}
y_x^{'} = \frac{y_z^{'}}{2\sqrt{x}}\\
y_x^{''} = \frac{y_z^{''}}{4x} - \frac{y_z^{'}}{4x\sqrt{x}}
\end{align*}

Putting these two equations into \textbf{\eqref{3s3}}:

\begin{align*}
z^2\left[\frac{y_z^{''}}{4x} - \frac{y_z^{'}}{4x\sqrt{x}}\right] +  \frac{y_z^{'}}{2\sqrt{x}} + \frac{1}{4} y = 0\\
y^{''} + \frac{y^{'}}{2z} + y=0\\
z^2y^{''} + y^{'}z + yz^2=0
\end{align*}

This is clearly in the format $x^2y^{''}+xy^{'}+(x^2-v^2)y=0$.

$$y(x) = c_1 J_0 (\sqrt{x}) + c_2 Y_0 (\sqrt{x})$$
\par
\subsubsection*{3.S.8}

For \textbf{\eqref{3s8}}, we allow $2x+1=z \therefore x=\frac{z-1}{2}$

\begin{align*}
y_x^{'} = 2y_z^{'}\\
y_x^{''} = 4y_z^{''}
\end{align*}

We then find $z^2(y^{''})+zy^{'}+(z^2-1)y = 0$.
Clearly, based on our format of Bessel's ODE, we can determine the solution.

$$y(x) = c_1 J_1 (2x+1) + c_2 Y_1 (2x+1)$$.
\par
\subsubsection*{3.S.10}

For \textbf{\eqref{3s10}}, we have 
\begin{align*}
y &= x^vy_z\\
y^{'} &= x^vy_z^{'}+vx^{v-1}y_z\\
y^{''} &= x^vy_z^{''} + 2vx^{v-1}y_z^{'}+(v-1)vy_zx^{v-2}
\end{align*}

Placing these into our equation, we obtain

\begin{align*}
x^{v+2}y^{''} + 2vx^{v+1}y^{'} + (v-1)vyx^{v} = 0\\\\
\left[x^{v+2}y^{''}\right] & \\
+ &\left[2vx^{v+1}y^{'} + x^{v+1}y^{'} - 2vx^{v+1}y^{'} \right]\\
+ &\left[v^2yx^v - yvx^v + vx^vy - 2v^2x^vy + v^2x^{2v}x^vy + v^2x^vy - v^4x^vy \right] = 0
\end{align*}

Using simple algebra and factorizing, we obtain
\begin{align*}
\left[x^vx^2\right]y^{''}+y^{'}\left[xx^v\right]+y\left[v^2x^2vx^v-v^2v^2x^v\right] = 0\\
x^2y^{''}+xy^{'}+(x^{2v}-v^2)6v^2=0
\end{align*}

Allowing for $v=1$, we obtain $x^2y^{''}+xy^{'}+(x^2-1)y=0$

Thus,
\begin{equation*}
y_z(x) = c_1 J_1(x) + c_2 Y_1(x)
\end{equation*}
Given that $y(x) = x^v y_z$ we find
\begin{equation*}
y(x) = c_1 x J_1(x) + c_2 x Y_1(x)
\end{equation*}


\end{document}
