\documentclass{article}
\usepackage[utf8]{inputenc}
\usepackage[english]{babel}
\usepackage{maplestd2e}
\usepackage[]{amsthm}
\usepackage[]{amssymb}
\usepackage[]{amsmath}
\usepackage[]{hyperref}
\usepackage[]{cancel}
\usepackage[]{graphicx}
\usepackage[]{xcolor}
\usepackage[]{tabularx}
\usepackage[]{fancyhdr}
\hypersetup{
    colorlinks,
    linkcolor={red!50!black},
    citecolor={blue!50!black},
    urlcolor={blue!80!black}
}

\pagestyle{fancy}
\fancyhf{}
\title{\vspace{-4cm}MTH30002 - Differential Equations Assignment 9}
\author{Joshua Rogers}
\lhead{MTH30002 Assignment 9}
\rhead{Joshua Rogers 101096819}
\date\today

\begin{document}
\maketitle
\section*{Exercise 4.4}

\begin{equation}\label{1}
\frac{du}{dt} = c^2 \frac{du^2}{dx^2}
\end{equation}

Instead of $c^2$ we use $k$, as it is easier to keep track of.

$$
u(x,0) = f(x)
$$

Using separation of variables, we assume
\begin{equation}\label{2}
u(x,t) = T(x)G(t)
\end{equation}
Placing this into $\eqref{1}$ and assuming a separabble constant of $\lambda$ we obtain
\begin{align*}
&G^{'} + k\lambda G = 0\\
&T^{''} + \lambda T =0
\end{align*}

Noting that $T(0)=0$ and $T(L)=0$, we solve for $T(x)$.

Solving for $\lambda=0$

\begin{align*}
T^{'} &= \int 0 dx =  A \\
T &= \int \int T^{''} dx = \int \int 0 dx = Ax+B
\end{align*}

$T(0) = 0 =B \therefore B = 0$.


$T(L) = AL + B = 0 \therefore A =0$.


Thus, $\lambda =0$ is not an eigenvalue.


For $\lambda <0$, we let $\lambda = -k^2$
$$
T''-k^2T=0
$$

$$T(x) = Ae^{kx} + Be^{-kx}$$

Using the boundary conditions, we realize that 
$$\det \left( \begin{bmatrix} 1 & 1 \\ e^{kL} & e^{-kL} \end{bmatrix} \right) \neq 0 \Bigr| k \neq 0 $$

Thus $\lambda <0$ does not give us any eigenfunctions.


For $\lambda > 0$ we let $\lambda = k^2$

$$ T''+k^2T=0$$

thus

$$T(x) = Acos(kx)+Bsin(kx)$$

$T(0)=0$ and $T(L)=0$:
$$\det \left( \begin{bmatrix} 1 & 0 \\ cos(kL) & sin(kL) \end{bmatrix} \right) = 0$$

We have

$$sin(kL) =0$$ thus $$k=\frac{\pi n}{L}$$

We have solved $$ \lambda_n = \frac{\pi^2n^2}{L^2} $$

\begin{equation}\label{A}
T_n(x) = sin(\frac{\pi n x}{L}) \Bigr| n \geq 1
\end{equation}


Solving for function $G$ we note


$$G^{''}+k\lambda_nG = 0$$

thus

\begin{equation}\label{B}
G(t) = Ae^{-k\lambda_nt}
\end{equation}

thus using $\eqref{2}$, $\eqref{A}$, and $\eqref{B}$ we obtain


$$
u_n(x,t) = G(t) \cdot T(x) = C_n e^{-t\left(\frac{c \pi n}{L}\right)^2} \cdot sin\left(\frac{\pi n x}{L}\right) \Bigr| n \geq 1
$$


We have now found $u_n(x,t)$.

This is clearly the solution to the odd expansion of a function $f(t)$. Using the principle of superposition, we have the solution:

$$u(x,t) = \sum_{n=1}^{\infty} C_n e^{-t\left(\frac{c \pi n}{L}\right)^2} \cdot sin\left(\frac{\pi n x}{L}\right) $$

where

$$C_n = \frac{2}{L} \int_{0}^{L} sin(\pi n t)f(t)dt$$



Finally, we find the result for the given information.

$$C_n = \frac{2}{10} \int_{0}^{5} 0.2t \cdot sin\left(\frac{n \pi t}{10}\right) dt + \frac{2}{10} \int_{5}^{10} 0 dt$$
Using integration by parts, obtain
$$C_n = \frac{-2cos\left(\frac{\pi n}{2}\right)}{\pi n} + \frac{4sin\left(\frac{\pi n}{2}\right)}{\pi^2 n^2}$$

The first part of $C_n$ we only want $C_{2n}$ as the other terms are zero. For the second part, we want $C_{2n-1}$.

Thus
$$C_{2n} = \frac{-2 (-1)^n}{2\pi n}$$

$$C_{2n-1} = \frac{ 4(-1)^{n+1}}{\pi^2(2n-1)^2}$$

Putting this all together, we obtain

$$u(x,t) = \sum_{n=1}^{\infty} \left[ \frac{-2(-1)^n}{2\pi n} e^{-t\left(\frac{2kn\pi}{L}\right)^2} sin\left(\frac{2xn\pi}{L}\right) + \frac{4\left(-1\right)^{n+1}}{\pi^2 (2n-1)^2} e^{-t\left(\frac{(2n-1)k\pi)^2}{L}\right)} sin\left(\frac{(2n-1)x\pi}{L}\right)\right]$$

Noting that $k=\frac{K}{cp} = \frac{1.04}{0.056 \cdot 10.6}$ we graph this on \href{https://www.desmos.com/calculator/aoywxidecv}{desmos}. Clearly, the temperature drops very quickly and the heat moves from the warmer side to the colder side, while also getting cooler at the ends due to the 0C endpoints (losing heat).

\section*{Exercise 4.5}

Here, we have $T^{'}(0)=0$ and $T^{'}(L)=0$. This is indicative of a fully insulated metal rod that retains heat.

Using the solution from the previous example, we have

Using separation of variables, we assume
\begin{equation}\label{3}
u(x,t) = T(x)G(t)
\end{equation}
Placing this into $\eqref{1}$ and assuming a separabble constant of $\lambda$ we obtain
\begin{align*}
&G^{'} + k\lambda G = 0\\
&T^{''} + \lambda T =0
\end{align*}

We once again solve for $T(x)$, however with $T^{'}(0)=0$ and $T^{'}(L)=0$.

For $\lambda = 0$

$$\int T^{''} dx = \int T^{'}(x) dx  = Ax +B$$
$$T^{'}(0) = A = 0$$
$$T^{'}(L) = A = 0$$

Thus $T_0(x) = 1 \cdot C_n\Bigr|n=0$ is an eigenfunction.


For $\lambda > 0$ we let $\lambda = k^2$
$$T(x) = Acos(k x) + Bsin(k x)$$
$$T^{'}(x) = -kAsin(kx)+kBcos(kx)$$

$$T^{'}(0) = -kAsin(0)+Bkcos(0) = Bk = 0 = B$$
$$T^{'}(L) = -kAsin(kL)+Bkcos(kL) = 0$$

$$\det \left( \begin{bmatrix} 0 & 1 \\ -ksin(kL) & kcos(kL) \end{bmatrix} \right) = 0 \Bigr| k \neq 0 $$

$$kL=\pi n$$

Thus we have found our eigenvalue and eigenfunction.
$$\lambda_n = \left(\frac{n \pi}{L}\right)^2$$
$$T_n(x) = cos\left(\frac{n \pi x}{L}\right) \Bigr| n \geq 1$$


Finally for $\lambda < 0$ we let $\lambda = -k^2$.

$$T(x) = Ae^{kx} + Be^{-kx}$$

$$T'(x) = kAe^{kx} - kBe^{-kx}$$
$$T^{'}(0) = kA - kB = A-B = 0 \therefore A=B$$

Thus there are no eigenfunctions to be found for this case.

We have from $\eqref{2}$ the function $G(T)$:
$$G(t) = Ae^{-k\lambda_nt}$$

Thus we have

$$
u_n(x,t) = G(t) \cdot T(x) = C_n e^{-t\left(\frac{c \pi n}{L}\right)^2} \cdot cos\left(\frac{\pi n x}{L}\right) \Bigr| n \geq 0
$$

For $n=0$ this will simply be equal to $C_n$.

This is clearly the even expansion of a function, $f(t)$ where $C_0 = \frac{1}{L} \int_{0}^{L} f(t) dt$.

Using the superposition principle,

$$u(x,t) = C_0 + \sum_{n=1}^{\infty} C_n e^{-t\left(\frac{c \pi n}{L}\right)^2} \cdot cos\left(\frac{\pi n x}{L}\right) $$


$C_o$ here represents the initial heat radiation.



\section*{Exercise 4.6}

We have $L=\pi$, $f(x)=1$, $c=1$.

$$
A_0 = \frac{1}{\pi} \int_0^{\pi} 1dx = 1
$$

$$
A_n = \frac{2}{\pi} \int_0^{\pi} cos\left(\frac{n \pi x}{pi}\right) = \frac{2sin(n \pi)}{\pi n} = 0\Bigr| x \in R
$$

Thus we have

$$
u(x,t) = 1
$$

This rod is fully insulated, and thus, the temperature is constant, as expected; i.e. no loss and no gain.

\section*{Exercise 4.7}

\begin{equation}\label{10}
V_t = c^2 V_xx- \beta v
\end{equation}

\begin{align}
&v(x,t) = u(x,t)w(t) \label{14}\\
&v_t = u_tw(t) + uw^{'}(t) \label{12} \\
&v_{xx} = u_{xx}w(t) \label{13}
\end{align}

Equating $\eqref{10}$ with $\eqref{12}$ and $\eqref{13}$

\begin{align*}
u_t w(t) + u w^{'} (t) = c^2u_{xx} w(t) - \beta u w \\
u_t w - c^2 u_{xx}  + \beta u w = -uw^{'} \\
\frac{w^{'}}{w} = \frac{u_t -c^2 u_{xx} + \beta u}{-u} \\
\frac{w^{'}}{w} = -\beta - \frac{u_t-c^2u_{xx}}{-u}
\end{align*}

Thus

$$
w^{'} = -\beta w
$$
which solves to
$$
e^{-\beta t}
$$
Using equation $\eqref{14}$, we obtain

$$v(x,t) = u(x,t) \cdot e^{-\beta t}$$

This shows that heat decays exponentially to a steady solution. Knowledge of initial conditions are lost, and we cannot 'tell the past temperature'

\end{document}
