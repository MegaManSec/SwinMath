\documentclass{article}
\usepackage[utf8]{inputenc}
\usepackage[english]{babel}
\usepackage{maplestd2e}
\usepackage[]{amsthm}
\usepackage[]{amssymb}
\usepackage[]{amsmath}
\usepackage[]{hyperref}
\usepackage[]{cancel}
\usepackage[]{graphicx}
\usepackage[]{xcolor}
\usepackage[]{tabularx}
\usepackage[]{fancyhdr}
\hypersetup{
    colorlinks,
    linkcolor={red!50!black},
    citecolor={blue!50!black},
    urlcolor={blue!80!black}
}

\pagestyle{fancy}
\fancyhf{}
\title{\vspace{-4cm}MTH30002 - Differential Equations Assignment 9}
\author{Joshua Rogers}
\lhead{MTH30002 Assignment 9}
\rhead{Joshua Rogers 101096819}
\date\today

\begin{document}
\maketitle

\section*{Exercise 4.4}
We have the following: $L=10cm$, $p = 10.6 gm \cdot cm^{-3}$, $k=1.04 cal/cm S^oC$, $\sigma = 0.056 cal/gm^oc$.

and

$u(0,t) = 0$, $u(10,t)=0$.

Noting that $f(x)$ can be odd-extended, and then taking the fourier series times the exponential decay we have

$$u(x,t) = \sum_{n=1}^{\infty} B_n sin\left( \frac{\pi n x}{L}\right) e^{-\frac{k n^2 \pi^2 t}{p \sigma L^2}}$$

$$B_n = \frac{2}{L} \int_0^L f(x) sin\left(\frac{n \pi x}{L}\right) dx$$

Plugging all our values in, we obtain

Using integration by parts
$$
\frac{2}{10} \int_0^{5} 0.2x \cdot sin\left(\frac{n \pi x}{10}\right)dx + \frac{2}{10} \int_5^{10} 0 dx = - \frac{2\left(\pi n cos(\frac{\pi n}{2}) -2sin(\frac{\pi n}{2})\right)}{\pi^2n^2}
$$

Thus we have
$$y=\left\{0\le x\le10:\sum_{n=1}^{\infty}-\frac{2\left(\pi n\cos\left(\frac{\pi n}{2}\right)-2\sin\left(\frac{\pi n}{2}\right)\right)}{\pi^{2}n^{2}}\sin\left(\frac{n\pi x}{10}\right)\cdot e^{\left(-1.04\cdot n^{2}\cdot\pi^{2}\cdot\frac{t}{10.6\cdot0.056\cdot10^{2}}\right)}\right\}$$

This can be graphed on \href{https://www.desmos.com/calculator/39jwqygv96}{desmos}. We can see that very quickly, the heat radiates towards the colder side of the bar, at the same time as being affected by the 0 degree endpoints.

\section*{Exercise 4.5}

$$
u_x(0,t) = u_x(L,t) =0.
$$

\begin{equation}\label{1}
\frac{du}{dt} = c^2 \frac{du^2}{dx^2}
\end{equation}
$$
u(x,0) = f(x)
$$

Using separation of variables, we assume
\begin{equation}\label{2}
u(x,t) = X(x)T(t)
\end{equation}
Placing this into $\eqref{1}$ we obtain
$$
\frac{T^{'}}{c^2T} = \frac{x^{''}}{x}
$$

or rather

$$
T^{'}x - c^2X^{''}T=0
$$
Let a separable constant be $-\lambda$

\begin{align*}
&T^{'} + \lambda c^2 T =0 \\
&X^{''} + \lambda X=0
\end{align*}

Solving for $\lambda=0$

\begin{align*}
X^{'} &= \int 0 dx =  A \\
X &= \int \int X^{''} dx = \int \int 0 dx = Ax+B
\end{align*}

From our original conditions we know that $X'(0) =0$ and $X'(L)=0$
Thus we $0=A$, thus $X=B$. is an eigen function. 


For $\lambda <0$, we let $\lambda = -k^2$
$$
X''-k^2X=0
$$
thus
$$X = C_1e^{kx} + C_2e^{-kx}$$
Using the boundary conditions $x^{'}(0)=0$ and $x^{'}(L)=0$ we determine for $k \neq 0$

$$\det \left( \begin{bmatrix} 1 & 1 \\ e^{kL} & e^{-kL} \end{bmatrix} \right) \neq 0$$

Thus $\lambda <0$ does not give us any eigenfunctions.

for $\lambda > 0$ we let $\lambda = k^2$

$$ X''+k^2X=0$$

thus

$$X = Acos(kx)+Bsin(kx)$$

$X'(0)=0$ and $X'(L)=0$:
$$\det \left( \begin{bmatrix} 1 & 0 \\ cos(kL) & sin(kL) \end{bmatrix} \right) = 0$$

We have

$$sin(kL) =0$$ thus $$k=\frac{\pi n}{L}$$

We have solved $$ \lambda = \frac{\pi^2n^2}{L^2} $$

\begin{equation}\label{A}
X_n = cos(\frac{\pi n x}{L})
\end{equation}

The sin terms cancel out.

Solving for function $T$ we note

$$\frac{T^{'}}{T} + \frac{\pi^2n^2c^2}{L} = 0$$

\begin{equation}\label{B}
T_n = \cdot e^{-\frac{\pi^2 n^2 c^2}{L^2} t}
\end{equation}

thus using $\eqref{2}$, $\eqref{A}$, and $\eqref{B}$ we obtain


$$
u_n(x,t) = T(t) \cdot X(x) = C_n cos(\frac{\pi n x}{L}) \cdot e^{-\frac{\pi^2 n^2 c^2}{L^2} t}
$$

Clearly, we have $u_0(x,t) = e^0 \cdot cos(0) \cdot C_0 = C_0$.

Thus

$$ u(x,t) = C_0 + \sum_{n=1}^{\infty} u_n(x,t) \cdot C_n $$

Likewise,

$$ u(0,t) = f(x) = C_0 + \sum_{n=1}^{\infty} C_n cos\left(\frac{\pi n x}{L} \right) \cdot e^0$$

This is clearly the even expansion of $f(x)$ on $[0,L]$ where $C_0 = \frac{1}{L} \int_0^L f(t)dt$ and $C_n = \frac{2}{L} \int_0^{L} f(t) cos\left(\frac{\pi n t}{L}\right) dt$

Thus, we have shown

$$
u(x,t) = \frac{1}{L} \int_0^L f(t) dt + \sum_{n=1}^{\infty} \frac {2}{L} \left[ \int_0^L f(t) cos\left(\frac{t \pi n}{L}\right)  dt \right] cos\left(\frac{t \pi n}{L}\right) e^{-\frac{\pi^2 n^2 c^2}{L^2} t}
$$



\section*{Exercise 4.6}

We have $L=\pi$, $f(x)=1$, $c=1$.

$$
A_0 = \frac{1}{\pi} \int_0^{\pi} 1dx = 1
$$

$$
A_n = \frac{2}{\pi} \int_0^{\pi} cos\left(\frac{n \pi x}{pi}\right) = \frac{2sin(n \pi)}{\pi n} = 0\Bigr| x \in R
$$

Thus we have

$$
u(x,t) = 1
$$

This rod is fully insulated, and thus, the temperature is constant.

\section*{Exercise 4.7}

\begin{equation}\label{10}
V_t = c^2 V_xx- \beta v
\end{equation}

\begin{align}
&v(x,t) = u(x,t)w(t) \label{14}\\
&v_t = u_tw(t) + uw^{'}(t) \label{12} \\
&v_{xx} = u_{xx}w(t) \label{13}
\end{align}

Equating $\eqref{10}$ with $\eqref{12}$ and $\eqref{13}$

\begin{align*}
u_t w(t) + u w^{'} (t) = c^2u_{xx} w(t) - \beta u w \\
u_t w - c^2 u_{xx}  + \beta u w = -uw^{'} \\
\frac{w^{'}}{w} = \frac{u_t -c^2 u_{xx} + \beta u}{-u} \\
\frac{w^{'}}{w} = -\beta - \frac{u_t-c^2u_{xx}}{-u}
\end{align*}

Thus

$$
w^{'} = -\beta w
$$
which solves to
$$
e^{-\beta t}
$$
Using equation $\eqref{14}$, we obtain

$$v(x,t) = u(x,t) \cdot e^{-\beta t}$$

This shows that heat decays exponentially to a steady solution. Knowledge of initial conditions are lost, and we cannot 'tell the past temperature'


\end{document}
