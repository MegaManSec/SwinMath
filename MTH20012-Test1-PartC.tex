\documentclass{article}
\usepackage[utf8]{inputenc}
\usepackage[english]{babel}
\usepackage[]{amsthm}
\usepackage[]{amssymb}
\usepackage[]{amsmath}
\usepackage[]{hyperref}
\usepackage[]{xcolor}
\usepackage[]{cancel}
\usepackage[]{fancyhdr}
\hypersetup{
    colorlinks,
    linkcolor={red!50!black},
    citecolor={blue!50!black},
    urlcolor={blue!80!black}
}

\pagestyle{fancy}
\fancyhf{}
\title{\vspace{-4cm}MTH20012 - Series and Transformations Test 1}
\author{Joshua Rogers}
\lhead{MTH20012 Test 1}
\rhead{Joshua Rogers 101096819}
\date\today

\begin{document}
\maketitle 

\section*{Part C}
\subsection*{Eleven}

Determine for which values of $x$ the following series is convergent, and which values it is absolutely convergent.

\begin{equation}\label{c11}
\sum_{n=1}^{\infty} (-1)^n \frac{e^{-4n-2}}{11n} x^n
\end{equation}

\subsubsection*{Answers}

Using the ratio test, we can determine for what values of $r$ the series is absolutely convergent.

\begin{align*}
\lim_{n\to\infty} \left| \frac{(-1)^{n+1}e^{-4n-6}x^{n+1}}{11(n+1)} \cdot \frac{11n}{(-1)^ne^{-4n-2}x^n}\right| \\
\lim_{n\to\infty} \left| \frac{(-1)\cancel{(-1)^n}e^{-4}\cancel{e^{-4n-2}}x\cancel{x^{n}}}{\cancel{11}(n+1)} \cdot \frac{\cancel{11}n}{\cancel{(-1)^n}\cancel{e^{-4n-2}}\cancel{x^n}}\right|\\
&\lim_{n\to\infty} \left| \frac{\cancel{n}(-e^{-4}x)}{\cancel{n}(1+\cancelto{0}{\frac{1}{n}})}\right| = \left|-e^{-4}x\right|
\end{align*}

Given that the series is absolutely convergent for $R < 1$, we find that $-e^4 < x < e^4$.

Letting $x = -e^4$, we find
\begin{align*}
\sum_{n=1}^{\infty} \frac{(-1)^n\cancel{e^{-4n}}e^{-2}}{11n}\cdot -\cancel{e^{4n}} = \frac{-1}{11e^2} \sum_{n=1}^{\infty} \frac{(-1)^n}{n}
\end{align*}

Noting that the harmoic series $\frac{(-1)^n}{n}$ converges, thus does the series \ref{c11} at $x = -e^4$.

Letting $x = e^4$ we find $\frac{1}{11e^2} \sum_{n=1}^{\infty} \frac{(-1)^n}{n}$ and thus the harmoic series is seen again.

Thus, the series converges for $e^4 \geq x \geq -e^4$.
\par

\subsection*{Twelve}

Determine the disk of convergence of the following power series

\begin{equation}\label{c12}
\sum_{n=0}^{\infty} \frac {(n+1)!(n+4)!(n+9)!}{(3n+1)!}z^n
\end{equation}

\subsubsection*{Answers}

Performing the ratio test,

\begin{align*}
\lim_{n\to\infty}\left| \frac{(n+2)!(n+5)!(n+10)! \cdot z \cdot \cancel{z^n}}{(3n+4)!} \frac{(3n+1)!}{(n+1)!(n+4)!(n+9)! \cdot \cancel{z^n}} \right| \\
= \lim_{n\to\infty}\left| \frac{z\cdot (n+2)(n+5)(n+10)}{3(n+1)(3n+2)(3n+4)}\right| \\
= \frac{z}{3} \lim_{n\to\infty}\left| \frac{(n+2)(n+5)(n+10)}{(n+1)(3n+2)(3n+4)} \right|\\
= \frac{z}{3} \lim_{n\to\infty}\left| \frac{\cancel{n^3}\left(1+\cancelto{0}{\frac{17}{n}+\frac{80}{n^2}+\frac{100}{n^3}} \right) }{\cancel{n^3}\left(9+\cancelto{0}{\frac{27}{n}+\frac{26}{n^2}+\frac{8}{n^3}}\right)}\right| \\
= \left|\frac{z}{27}\right|
\end{align*}

We have $\left|\frac{z}{27}\right| < 1$ and thus $\left|z\right| < 27$.

The \textit{radius} is therefore $27$.
The disk of convergence may include the endpoints. Do they? No. How do I know? using the comparison test(the compare-it-on-wolframalpha-test), the series is seen to be divergent in the case that $z = 27$ and $z = -27$. Do I expect any mark for this unsubstantiated claim? No!

Therefore we find the disk of convergence. 

$ |z - z_0| < 27 $ in this case $z_0 = 0$ thus $|z| <27$ centered at $x_0=0$.
\par

\subsection*{Thirteen}

Determine the disc of convergence of the following power series $f(z)$, and thus, obtain a closed form expression for the power series $f(z) \cdot n^2$

\begin{equation}\label{c13}
\sum_{n=0}^{\infty} (-1)^n2^{n-1}15z^n
\end{equation}


\subsubsection*{Answers}
Root test.

\begin{align*}
\lim_{n\to\infty} \left|\left( (-1)^n \cdot 2^{-1} \cdot 2^n \cdot 15 \cdot z^n \right)^{1/n}\right|\\
\lim_{n\to\infty} \left|\left( (-1) \cdot \cancelto{2}{2^{\frac{-1}{n}}} \cdot \cancelto{1}{2^n} \cdot \cancelto{1}{15} \cdot \cancelto{z}{z^n} \right)\right| = \left|-2z\right| \\
\therefore \left|z\right| < \frac{1}{2}
\end{align*}

The disk of convergence is therefore $\left|z\right| < \frac{1}{2}$.

Finding a closed form expression, we identify that:
\begin{align*}
\sum_{n=0}^{\infty}  (-1)^n \cdot 2^{-1} \cdot 2^n \cdot 15 \cdot z^n = \frac{15}{2} \sum_{n=0}^{\infty} (-2z)^n
\end{align*}
This is a geometric series with $r = -2z$. Thus, our closed form expression is
\[
f(z) = \frac{15}{2(2z+1)} \Bigr| \left|z\right| < \frac{1}{2}
\]

Finding a closed form expression for $f(z) \cdot n^2$ will not be fun.

Nonetheless.

We have
\[
f(z) = \frac{1}{1+2z} \quad f^{'}(z) = \frac{-2}{(1+2z)^2} \quad f^{''}(z) = \frac{8}{(1-2z)^3}
\]

\[
\frac{d}{dz} \sum_{n=0}^{\infty} (-2z)^n = \sum -2n(-2z)^{n-1} = \frac{-1}{z} \sum n(-2z)^n
\]

\[
z \cdot \frac{d}{dz} \sum (-2z)^n = \sum n(-2z)^n
\]

Re-arranging, we find

\[
\sum n(-2z)^n = \frac{-2z}{(1+2z)^2}
\]

This is close to what we want, but not quite.

\begin{align*}
\frac{d^2}{dz^2} \sum (-2z)^n = \frac{d}{dz} \sum -2n \cdot (-2z)^{n-1} = \sum 4n(n-1)(-2z)^{n-2} = \frac{4}{(-2z)^2} \sum 4n(n-1)(-2z)^n\\
= \frac{4}{4z^2} \left[ \sum n^2(-2z)^n - \sum n(-2z)^n\right]\\
= \frac{1}{2} \left[ \sum n^2(-2z)^n + \frac{2z}{(1+2z)^2}\right]
\end{align*}

Equating thee terms and doing some algebra we find
\begin{align*}
z^2 \cdot \frac{d^2}{dz^2} \sum (-2z)^n = \sum n^2(-2z)^n + \frac{2z}{(1+2z)^2}\\
\therefore \sum_{n=0}^{\infty} n^2 \cdot (-2z)^n = \frac{-2z}{(1+2z)^2} + \frac{8z^2}{(1+2z)^3} = \frac{2z(2z-1)}{(2z+1)^3}
\end{align*}

Finally, we find

\[
f(z) \cdot n^2 = \frac{15z(2z-1)}{(2z+1)^3} \Bigr| \left|z\right| < \frac{1}{2}
\]

Double checking this result, we find that with $z = \frac{1}{3}$, $\sum (-1)^n \cdot 2^{n-1} \cdot 15 \cdot \frac{1}{3}^n \cdot n^2 = \frac{-9}{25}$.
Likewise we find that $f(\frac{1}{3}) \cdot n^2 = \frac{-9}{25}$, confirming the two equations as the same.

\end{document}
