\documentclass{article}
\usepackage[utf8]{inputenc}
\usepackage[english]{babel}
\usepackage[]{amsthm}
\usepackage[]{amssymb}
\usepackage[]{amsmath}
\usepackage[]{hyperref}
\usepackage[]{xcolor}
\usepackage[]{cancel}
\usepackage[]{fancyhdr}
\hypersetup{
    colorlinks,
    linkcolor={red!50!black},
    citecolor={blue!50!black},
    urlcolor={blue!80!black}
}

\pagestyle{fancy}
\fancyhf{}
\title{\vspace{-4cm}MTH20012 - Series and Transformations Test 1}
\author{Joshua Rogers}
\lhead{MTH20012 Test 1}
\rhead{Joshua Rogers 101096819}
\date\today

\begin{document}
\maketitle 

\section*{Part B}
\subsection*{Seven}

Verify that the integral test can be applied to the following series, and use the integral test to determine if the series converges.

\begin{equation}\label{b7}
\sum_{n=3}^{\infty} \frac{\ln{(n+2)}^5}{n+2}
\end{equation}

\subsubsection*{Answers}

To utilize the integral test, the series must be definable as $f(x) = \ref{b7}$ and thus be: positive, continuous, and decreasing.

For $f(x) = \frac{\ln{x+2}^5}{x+2} [3,\infty)$, it is clear $f(x)$ is always positive within our domain as $\ln$ will only return positive results.

Likewise it is decreasing as for sufficiently large $x$ the denominator dominates (the local extrema is $x=e^5-2$).

Finally, it is continuous within the domain we use.

Now, we use the integral test.

\[
\int_3^\infty \frac {\ln(n+2)^5}{n+2}dn = \frac{\ln(n+2)^6}{6}\Bigr|_{n=3}^{n=\infty}
\]

Evaluating, we find
\[
\lim_{n\to\infty}\left(\frac{\ln(n)}{6}\right) - \frac{\ln(9)}{9} = \infty.
\]

Thus, the series does not converge.

\par

\subsection*{Eight}

Verify that the comparison test can be applied to the following series, and use the integral test to determine if the series converges.

\begin{equation}\label{b8}
\sum_{n=1}^{\infty} \frac{17\sqrt{n}-4}{3n^2+8n+11}
\end{equation}

\subsubsection*{Answers}

Clearly, for $n \to \infty$, the series \ref{b8} behaves like the series with only the highest powers of $k$, i.e. for sufficiently large $k$, we can say that \ref{b8} behaves like $\frac{17\sqrt{n}}{3n^2} = \frac{17}{n^{\frac{3}{2}}}$

\begin{align*}
a_n = \frac{17\sqrt{n}-4}{3n^2+8n+11} \quad b_n = \frac{17}{3n^\frac{3}{2}}
\end{align*}

Using the limit comparison test:

\begin{align*}
lim_{n\to\infty} \frac{a_n}{b_n} = lim_{n\to\infty} \frac{17\sqrt{n}-4}{3n^2+8n+11} \cdot  \frac{3n^\frac{3}{2}}{17} = \frac{3(-4+17\sqrt{n})n^{\frac{3}{2}}}{17(11+8n+3n^2)}
= \frac{\cancel{3}\cancel{n^2}(\cancelto{0}{\frac{-4}{\sqrt{n}}}+\cancel{17})}{\cancel{17n^2}(\cancelto{0}{\frac{11}{n^2}}+\cancelto{0}{\frac{8}{n}}+\cancel{3})} = 1.
\end{align*}

Therefore, both $\sum_{n=1}^{\infty} a_n$ and  $\sum_{n=1}^{\infty} b_n$ are either mutually convergent or mutually divergent.
Given that $b_n$ is a harmonic-p series with $p=\frac{3}{2}$, we therefore declare that equation \ref{b8} is convergent.


\par

\subsection*{Nine}

Use the root test to determine whether the following series is absolutely convergent.

\begin{equation}\label{b9}
\sum_{n=1}^{\infty} \frac{(-1)^n 17^n}{3^{n+2} n^{n+3}}
\end{equation}

\subsubsection*{Answers}

We can rewrite the sum from equation \ref{b9} and conduct the root test.
\begin{align*}
\lim_{n\to\infty} \left|\left(\frac{(-1)^n 17^n}{3^2 3^n n^3 n^n}\right)\right|^{\frac{1}{n}}\\
= \lim_{n\to\infty} \left|\left(\frac{(-1) \cdot 17}{\cancelto{1}{3^{\frac{2}{n}}} \cdot 2 \cdot \cancelto{1}{n^{\frac{3}{n}}} \cdot \cancelto{\infty}{n}} \right) \right| = \frac{1}{\infty} = 0
\end{align*}

Clearly, $L = 0 < 1$. Therefore, the series is absolutely convergent.

\par

\subsection*{Ten}

Determine a closed form expression for the following series.

\begin{equation}\label{b10}
\sum_{n=4}^{\infty} \frac{21}{(3n+2)(3n+8)}
\end{equation}
\subsubsection*{Answers}

The aim of the game is to notice the telescopic series. Using partial fraction decomposition, we have

\begin{align*}
\frac{21}{(3n+2)(3n+8)} = \frac{7}{2}\left(\frac{1}{3n+2}-\frac{1}{3n+8}\right)
\end{align*}

'Walking up' the series, we find
\begin{align*}
\frac{7}{2}\left[\frac{1}{14} \cancel{- \frac{1}{20}} + \frac{1}{17} \cancel{- \frac{1}{23}} \cancel{+ \frac{1}{20}} \cancel{- \frac{1}{26}} \cancel{+ \frac{1}{23}} \cancel{- \frac{1}{29}} + \cdots \right]\\
= \frac{7}{2}\left(\frac{1}{14}+\frac{1}{17}\right) \cdots - \frac{7}{2}\left(\frac{1}{3n+2}\right)
\end{align*}

Taking the $\lim_{n\to\infty}$, we find the limit to be $ \frac{7}{2}\left(\frac{1}{14}+\frac{1}{17}\right) = \frac{31}{68}$.
\par

\end{document}
