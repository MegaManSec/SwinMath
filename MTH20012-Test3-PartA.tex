\documentclass{article}
\usepackage[utf8]{inputenc}
\usepackage[english]{babel}
\usepackage[]{amsthm}
\usepackage[]{amssymb}
\usepackage[]{amsmath}
\usepackage[]{graphicx}
\usepackage[]{hyperref}
\usepackage[]{xcolor}
\usepackage[]{cancel}
\usepackage[]{fancyhdr}
\hypersetup{
    colorlinks,
    linkcolor={red!50!black},
    citecolor={blue!50!black},
    urlcolor={blue!80!black}
}

\pagestyle{fancy}
\fancyhf{}
\title{\vspace{-4cm}MTH20012 - Series and Transformations Test 3}
\author{Joshua Rogers}
\lhead{MTH20012 Test 3}
\rhead{Joshua Rogers 101096819}
\date\today

\begin{document}
\maketitle 

\section*{Part A}
\subsection*{Question 1}
Find the Laplace transform of the following function:
$$f(t) = e^{3t}sin(4t)-te^{-3t}sin(t)$$
\subsubsection*{Answer}

Using linearity,
$$ \mathcal{L}\{f(t)\} = \mathcal{L}\{e^{3t} sin(4t)\} - \mathcal{L}\{t e^{-3t} sin(t)\}$$

For the first part we use the standard Laplace transform of $sin(at)$ where $a=4$. Likewise, using the first shift theorem with $a=3$ we obtain

$$\frac{4}{(s-3)^2+16}$$

For the second part we do the same thing, but noting that we also use the multiplcation theorm that $tF(t) = -f^{'}(s)$

$$-f^{'}(s) = \frac{2(s+3)}{\left((s+3)^2+1\right)^2}$$
i.e.
$$- \mathcal{L}\{e^{-3t} sin(t)\} = - \frac{2(s+3)}{\left((s+3)^2+1\right)^2}$$

Thus, we have have the laplace tansformation

$$\mathcal{L}\{f(t)\} = \frac{4}{(s-3)^2+16} - \frac{2(s+3)}{\left((s+3)^2+1\right)^2}$$

\subsection*{Question 2}
Find the laplace transform of the following function

\[ f(t) = \begin{cases}
     3t & 0\leq t < 2 \\
     0 & 2\leq t < 4 \\
     -3t^2 & 4 \leq t < 5 \\
     1 & 5 \leq t
   \end{cases}
\]

\subsubsection*{Answer}

Effectively, we want to solve the following integral

$$\int_{0}^{2} 3t e^{-st} dt  + \int_{4}^{5} -3t^2 e^{-st} dt + \int_{5}^{\infty} e^{-st} dt$$

I'm going to answer this question twice, just because I didn't want to bother with the Heaviside function and I prefer integrals, but I realize it may be required to show that I know how it works for full marks. However, having written out all of the integration, I couldn't have the heart to delete it.


Evaluating this step by step, we start with the final part of the equation.

\begin{equation}\label{last}
\int_{5}^{\infty} e^{-st} dt = \lim_{t \to \infty} \frac{e^{-st}}{s} - \lim_{t \to 5} \frac{e^{-st}}{s} = - \frac{e^{-5s}}{s}
\end{equation}

For the first part, we solve the indefinite integral first (more on this later.)

$$3\int t e^{-st} dt$$

Using integration by parts, we let $f=t$, $df=dt$, $dg=e^{-st}dt$, and $g=-\frac{e^{-st}}{s}$.

\begin{equation}\label{firstbase}
 = -\frac{3te^{-st}}{s}-\frac{3}{s^2}\int e^u du = -\frac{3e^{-st}(st+1)}{s^2}
\end{equation}

Between $t=0$ and $t=2$ this is simply

\begin{equation}\label{first}
\frac{3(1-e^{-2s}(2s+1))}{s^2}
\end{equation}


Finally, we must solve the middle part, using integration by parts twice.
We first attempt to solve the indefinite integral
$$\int t^2 e^{-st} dt$$
remembering that we must multiply the result by $-3$ later on.

We let $f=t^2$, $df=2tdt$, $dg=e^{-st}dt$, and $g=-\frac{e^{-st}}{s}$

thus we have

$$=-\frac{t^2e^{-st}}{s}+\frac{2}{s} \int t e^{-st} dt$$

And we have already solved this remaining integral already (not-withstanding the $3$ out of the front) in $\eqref{firstbase}$.

thus we have

$$-3 \int t^2 e^{-st} dt = -\frac{t^2e^{-st}}{s}+\frac{2}{s} \cdot \frac{-e^{-st}(st+1)}{s^2} = 3\frac{e^{-st}\left(s^2t^2+2st+2\right)}{s^3}$$

Between $t=4$ and $t=5$ this is

\begin{equation}\label{middle}
3\frac{e^{-5s}\left(25s^2+10s+2\right)}{s^3} - 3\frac{e^{-4s}\left(16s^2+8s+2\right)}{s^3}
\end{equation}

Thus, finally, we add up $\eqref{first}$, $\eqref{middle}$, and $\eqref{last}$ and obtain

$$
F(s) = \frac{3(1-e^{-2s}(2s+1))}{s^2} + \frac{3e^{-5s}\left(25s^2+10s+2\right) - 3e^{-4s}\left(16s^2+8s+2\right)}{s^3} - \frac{e^{-5s}}{s}
$$

Which, although may be ugly, is indeed the correct answer!

Alternatively, we do it a different way, using the heaviside step function.


$$f(t) = 3t \left[ H(t-0)-H(t-2)\right ] -3t^2 \left[ H(t-4) - H(t-5) \right] + 1 \left[ H(t-5) \right]$$
$$f(t) = 3tH(t) - 3tH(t-2) - 3t^2(H(t-4)) + 3t^2(H(t-5)) + H(t-5)$$

Noting that we have
$$ \mathcal{L} \left\{ H(t-c) f(t-c) \right\} = e^{-cs} F(s)$$
and
$$ \mathcal{L} \left\{H(t-c)\right\} = \frac{e^{-cs}}{s}$$

We split up the $f(t)$ into smaller $g(t)$s and proceed accordingly.

$$F(s) = \sum \mathcal{L}\{g(t)\}$$

\begin{align*}
&g(t) = 3tH(t) = 3t = \frac{3}{s^2} \\
&g(t) = -3tH(t-2) = (-3 (t-2)-6)H(t-2) =-3(t-2)H(t-2)-6H(t-2) \\
&= -\frac{3e^{-2s}}{s^2} - \frac{6e^{-2s}}{s} \\
& g(t) = -3t^2(H(t-4)) = -3((t-4)^2+8(t-4)+16)H(t-4) \\
& -3(t-4)^2H(t-4)-24(t-4)H(t-4)-48H(t-4) \\
&= -\frac{6e^{-4s}}{s^3}-\frac{24e^{-4s}}{s^2}-\frac{48e^{-4s}}{s} \\
&g(t) = 3t^2(H(t-5) =  3((t-5)^2+10(t-5)+25)H(t-5) \\
&= \frac{6e^{-5s}}{s^3}+\frac{30e^{-5s}}{s^2}+\frac{75e^{-5s}}{s} \\
&g(t) = H(t-5) = \frac{e^{-5s}}{s}
\end{align*}

Thus we have found

$$F(s) = \frac{3}{s^2} -\frac{3e^{-2s}}{s^2} - \frac{6e^{-2s}}{s} -\frac{6e^{-4s}}{s^3}-\frac{24e^{-4s}}{s^2}-\frac{48e^{-4s}}{s} + \frac{6e^{-5s}}{s^3}+\frac{30e^{-5s}}{s^2}+\frac{76e^{-5s}}{s}$$

which, is another representation of the function we found earlier. I still prefer integration.


\subsection*{Question 3}

Find the inverse Laplace transform of the following function:

$$F(s) = \frac{e^{-3s}\left(2s+1\right)}{s^2-4s+13}$$

\subsubsection*{Answer}
Using the inverse transform rule $\mathcal{L}^{-1}\left\{e^{-as}F\left(s\right)\right\}=H\left(t-a\right)f\left(t-a\right)$

we have $a=3$ and (then using partial fractions)
$$F(s) = \frac{2s+1}{s^2-4s+13} = 2 \frac{(s-2)}{\left(s-2\right)^2+9}+5\frac{1}{\left(s-2\right)^2+9}$$

The first part of this is clearly the cosine function with frequency $3$, shifted by $a=2$, and all multiplied by $2$.

The second part is similar but the sin function divided by $3$ and then multiplied by $5$

thus, we have

$$f(t) = 2\mathcal{L}^{-1}\left\{\frac{s-2}{\left(s-2\right)^2+9}\right\}+5\mathcal{L}^{-1}\left\{\frac{1}{\left(s-2\right)^2+9}\right\} = 2e^{2t}\cos \left(3t\right)+\frac{5}{3}e^{2t}\sin \left(3t\right)$$

Thus by our defition of the inverse transform rule, we have $f(t) = H(t-3)f(t-3)$:

$$f(t) = \text{H}\left(t-3\right)\left[2e^{2\left(t-3\right)}\cos \left(3\left(t-3\right)\right)+\frac{5}{3}e^{2\left(t-3\right)}\sin \left(3\left(t-3\right)\right)\right]$$

or to have it a bit nicer..

$$f(t) = \frac{1}{3} H(t-3) e^{2t-6} \left[6cos\left(3t-9\right)+5sin\left(3t-9\right)\right]$$
\end{document}
